\documentclass{article}\usepackage[]{graphicx}\usepackage[]{color}
%% maxwidth is the original width if it is less than linewidth
%% otherwise use linewidth (to make sure the graphics do not exceed the margin)
\makeatletter
\def\maxwidth{ %
  \ifdim\Gin@nat@width>\linewidth
    \linewidth
  \else
    \Gin@nat@width
  \fi
}
\makeatother

\definecolor{fgcolor}{rgb}{0.345, 0.345, 0.345}
\newcommand{\hlnum}[1]{\textcolor[rgb]{0.686,0.059,0.569}{#1}}%
\newcommand{\hlstr}[1]{\textcolor[rgb]{0.192,0.494,0.8}{#1}}%
\newcommand{\hlcom}[1]{\textcolor[rgb]{0.678,0.584,0.686}{\textit{#1}}}%
\newcommand{\hlopt}[1]{\textcolor[rgb]{0,0,0}{#1}}%
\newcommand{\hlstd}[1]{\textcolor[rgb]{0.345,0.345,0.345}{#1}}%
\newcommand{\hlkwa}[1]{\textcolor[rgb]{0.161,0.373,0.58}{\textbf{#1}}}%
\newcommand{\hlkwb}[1]{\textcolor[rgb]{0.69,0.353,0.396}{#1}}%
\newcommand{\hlkwc}[1]{\textcolor[rgb]{0.333,0.667,0.333}{#1}}%
\newcommand{\hlkwd}[1]{\textcolor[rgb]{0.737,0.353,0.396}{\textbf{#1}}}%
\let\hlipl\hlkwb

\usepackage{framed}
\makeatletter
\newenvironment{kframe}{%
 \def\at@end@of@kframe{}%
 \ifinner\ifhmode%
  \def\at@end@of@kframe{\end{minipage}}%
  \begin{minipage}{\columnwidth}%
 \fi\fi%
 \def\FrameCommand##1{\hskip\@totalleftmargin \hskip-\fboxsep
 \colorbox{shadecolor}{##1}\hskip-\fboxsep
     % There is no \\@totalrightmargin, so:
     \hskip-\linewidth \hskip-\@totalleftmargin \hskip\columnwidth}%
 \MakeFramed {\advance\hsize-\width
   \@totalleftmargin\z@ \linewidth\hsize
   \@setminipage}}%
 {\par\unskip\endMakeFramed%
 \at@end@of@kframe}
\makeatother

\definecolor{shadecolor}{rgb}{.97, .97, .97}
\definecolor{messagecolor}{rgb}{0, 0, 0}
\definecolor{warningcolor}{rgb}{1, 0, 1}
\definecolor{errorcolor}{rgb}{1, 0, 0}
\newenvironment{knitrout}{}{} % an empty environment to be redefined in TeX

\usepackage{alltt}
\usepackage{Sweave}
\usepackage{float}
\usepackage{graphicx}
\usepackage{tabularx}
\usepackage{siunitx}
\usepackage{mdframed}
\usepackage{natbib}
\bibliographystyle{..//papers/styles/besjournals.bst}
\usepackage[small]{caption}
\setkeys{Gin}{width=0.8\textwidth}
\setlength{\captionmargin}{30pt}
\setlength{\abovecaptionskip}{0pt}
\setlength{\belowcaptionskip}{10pt}
\topmargin -1.5cm        
\oddsidemargin -0.015cm   
\evensidemargin -0.015cm
\textwidth 16cm
\textheight 21cm 
%\pagestyle{empty} %comment if want page numbers
\parskip 7.2pt
\renewcommand{\baselinestretch}{1.5}
\parindent 0pt

\newmdenv[
  topline=true,
  bottomline=true,
  skipabove=\topsep,
  skipbelow=\topsep
]{siderules}
\IfFileExists{upquote.sty}{\usepackage{upquote}}{}
\begin{document}

\renewcommand{\thetable}{\arabic{table}}
\renewcommand{\thefigure}{\arabic{figure}}
\renewcommand{\labelitemi}{$-$}

\renewcommand{\thesection}{\arabic{section}.}
\renewcommand\thesubsection{\arabic{section}.\arabic{subsection}} 

%%%%%%%%%%%%%%%%%%%%%%%%%%%%%%%%%%%%
\section*{Field Freezing Experiment}

\textbf{ORIGINAL DESIGN (without a real false spring): } I will design a field experiment to evaluate the differences in damage sustained across life stage to assess forest recruitment and sustainability. I will monitor the phenology of sapling and adult individuals across 8-9 species for 16 adults and 24 saplings per species and expose half of the individuals to simulated false spring events (Table \ref{tab:fieldfrz}). I will quantify the traits important for frost tolerance (i.e. leaf serrations and number of trichomes) and monitor those traits across life stage.  To simulate a false spring, I will construct multiple in-field growth chambers and place them over individuals between budburst and leafout for an hour just after sunrise, which is the coldest time of day. I will then monitor their growth and phenology from budburst to leaf drop to determine the effects of false springs across life stage.

\textbf{QUESTIONS: }
\begin{enumerate}
\item Do different life stages utilize different avoidance and/or tolerance strategies?
\item Does false spring damage vary across life stage?
\end{enumerate}

\textbf{PLANS FOR ANALYSIS: }
\newline
\textit{Question 1: } I want to first evaluate the variation in strategies across life stage. For this question, I will have either Duration of Vegetative Risk, number of trichomes or number of leaf serrations as the response variable (Equation \ref{eq:1}).
\begin{equation} \label{eq:1}
y_i = \alpha_{sp_(i)} + \beta_{lifestage_{sp_{(i)}}} + \sigma_{sp_{(i)}}
\end{equation}

\textit{Question 2: } I will look at how the duration of vegetative risk shifts under the treatment and I will also have leaf chlorophyll content and SLA as response variables (Equation \ref{eq:2}).
\begin{equation} \label{eq:2}
y_i = \alpha_{(i)} + \beta_{tx_{sp_{(i)}}} + \beta_{serrations_{sp_{(i)}}} + \beta_{trichomes_{sp_{(i)}}} + \sigma_{sp_{(i)}}
\end{equation}

\textbf{POTENTIAL SHIFT IN DESIGN (if a real false spring is expected): } Could I place hobo loggers on individuals that are between budburst and leafout and if the temperature drops below -2.2$^{\circ}$C then it is considered a false spring treatment? The temperatures would vary but at least I would be monitoring that temperature...? Still thinking! Temperatures have been in the 50s and 60s over the last couple of weeks. We are supposed to get into the high 60s a couple times this week and then it's suppose to drop again... and then go back up to the 50s. I few of my smaller individuals are already showing signs of bud swelling. From my experiment last spring, I saw that -- depending on the species -- the buds could look like that for quite a while before bursting so it may be okay, but it is still so early in the year. My concern is that I will start the experiment and then a false spring will hit halfway through and I will greatly reduce my sample size and number of species. Right now, I will continue to monitor the buds and if I think an individual is very close to budburst, I will look at the 10 day forecast and try to determine the best course of action from there. I also have a lot of individuals tagged at the Grant (Table \ref{tab:grant}), which I will go check on those individuals in the next couple of weeks. Maybe one site will have a false spring and the other won't --- at least one can hope!


\begin{table}[H]
\centering
\caption{Number of individuals already tagged in Harvard Forest for the spring field season.}
\begin{tabular}{|c | c | c |}
\hline
\textbf{Species} & \textbf{Stage} & \textbf{\# of Individuals} \\
\hline
\textit{Acer pensylvanicum} & Sapling & 24 \\
\hline 
\textit{A. saccharum} & Sapling & 24 \\
\hline
\textit{Betula lenta} & Sapling & 24 \\
\hline
\textit{Carya ovata} & Sapling & 24 \\
\hline
\textit{Corylus cornuta} & Sapling & 24 \\
\hline
\textit{Fagus grandifolia} & Sapling & 16 \\
\hline
\textit{Hamamelis virginiana} & Sapling & 24 \\
\hline
\textit{Ilex verticillata} & Sapling & 24 \\
\hline
\textit{Viburnum acerfolium} & Sapling & 24 \\
\hline
\textit{A. pensylvanicum} & Tree & 16 \\
\hline
\textit{A. saccharum} & Tree & 16 \\
\hline
\textit{B. lenta} & Tree & 16 \\
\hline
\textit{C. ovata} & Tree & 16 \\
\hline
\textit{C. cornuta} & Tree & 16 \\
\hline
\textit{F. grandifolia} & Tree & 16 \\
\hline
\textit{H. virginiana} & Tree & 16 \\
\hline
\textit{I. verticillata} & Tree & 16 \\
\hline
\textit{V. acerfolium} & Tree & 16 \\
\hline
\end{tabular}
\label{tab:fieldfrz}
\end{table}

\begin{center}
\captionof{table}{Field Freezing (Grant) - number of saplings marked per species. } \label{tab:grant} 
\footnotesize
\begin{tabular}{|c | c |}
\hline
\textbf{Species} & \textbf{No. of Individs} \\
\hline
VIBLAN & 16 \\
\hline
ILEMUC & 16 \\
\hline
BETALL & 16 \\
\hline
FAGGRA & 16 \\
\hline
ALNINC & 16 \\
\hline
PRUPEN & 16 \\
\hline
ACERUB & 16 \\
\hline
ACESAC & 16 \\
\hline
ACEPEN & 16 \\
\hline
\end{tabular}
\end{center}



\end{document}
